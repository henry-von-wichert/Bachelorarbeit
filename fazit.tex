\chapter{Fazit und Ausblick}

In dieser Arbeit wurde die Sekundärelektronenemission an einer Mikroentladung und durch Energiestrommessungen untersucht. Insbesondere wurden die effektiven SEE Koeffizienten eines Atmosphärendruck-DC-Mikroplasmas in Helium auf verschiedenen Oberflächen bestimmt. 

Um an einem Mikroplasma Energiestrommessungen durchzuführen, wurde dieses zwischen zwei passiven Thermosonden gezündet, die als Elektroden dienten. Dafür wurden Thermosonden aus verschiedenen Materialien hergestellt und kalibriert. Es konnte gezeigt werden, das dies eine gute Methode ist, mit der nahezu der vollständige Energiestrom aus dem Plasma vermessen und auf Kathode und Anode aufgeteilt werden kann.\\

Es konnte gezeigt werden, dass die effektiven SEE Koeffizienten bei Atmosphärendruck deutlich größer sind als bei Niederdruck. Dabei ergaben sich Werte der Größenordnung 1, im Vergleich zu den Werten von \numrange{0,1}{0,3}, die aus Niederdruckplasmen bekannt sind \cite{liebermanPrinciplesPlasmaDischarges2005,haaseDynamicDeterminationSecondary2018,bohmRetardingfieldAnalyzerMeasurements1993,toliasSecondaryElectronEmission2014, marcakNoteIoninducedSecondary2015,dakshaComputationallyAssistedSpectroscopic2016,pamperinIoninducedSecondaryElectron2018,dakshaMaterialDependentModeling2019,oechsnerElectronYieldsClean1978,guntherschulzeElektronenablosungDurchStoss1930}. Die gemessenen Koeffizienten stimmen dabei mit anderen Messungen überein \cite{hansenConventionalNonconventionalDiagnostics2022}.\\

In den Energiestrommessungen zeigte sich eine deutliche und reproduzierbare Asymmetrie, bei der die Kathode stärker erhitzt wurde als die Anode. Diese ist wahrscheinlich durch den direkten und indirekten Energieeintrag der Ionen bedingt, die in der Randschicht auf die Kathode beschleunigt werden und schließlich zur SEE beitragen.

Die aus den thermischen und elektrischen Messungen bestimmten Wirkungsgrade zeigten sich mit etwa 90\% im Vergleich zu anderen Atmosphärendruckplasmen sehr hoch. Dies liegt daran, dass die Elektroden das sehr kleine Plasma räumlich fast völlig umschließen. So kann fast der vollständige Energiestrom aus dem Plasma gemessen werden, was dessen Messung aussagekräftiger macht.
\\

Das größte Hindernis bei der Messung am Plasma waren Instabilitäten wie Funkenentladungen bei der Zündung. Diese können die Zündung eines Plasmas vollständig verhindern, wenn sie zu stark auftreten. Schwächere Störungen ermöglichen zwar eine Zündung, können aber die serielle Kommunikation zu den Sonden durch induzierte Stromstöße in den Datenkabeln stören. Dieses Problem konnte jedoch bei Helium durch die galvanische Trennung der Auswertungskabel von den Sondenplatinen gelöst werden.

Diese Instabilitäten werden vor Allem durch Verunreinigungen in der Gasatmosphäre verursacht und können durch einen besseren Füllprozess der Kammer stark vermindert werden. Deshalb wurden Untersuchungen angestellt, wie sich eine reine Gasatmosphäre herstellen lässt. Da die Gasbox, in der das Plasma gezündet wird, gasdicht ist, kann sie zum Befüllen mehrfach abgepumpt und neu geflutet werden. So wird gewährleistet, dass keine nennenswerten Rückstände von Luft in der Box verbleiben.

Zur Prüfung der Gasreinheit wurden spektroskopische Messungen durchgeführt, die eine leichte Sauerstoffverunreinigung, aber keine Zeichen von Stickstoff zeigen. Dies zeigt, dass es bei sorgfältiger Befüllung keine Kontamination durch Außenluft gibt und die Gasbox dicht ist. Die Linienstärken des Spektrums bleiben im Zeitverlauf konstant, was auch gegen ein Leck in der Kammer spricht. Diese Messungen zeigen das für eine nicht-thermische Entladung erwartete Linienspektrum.\\

Die Messung von Strom-Spannungs-Kennlinien zeigte, dass die über dem Plasma abfallende Spannung mit höherem Entladungsstrom nicht steigt und bestätigte so die Erzeugung einer Glimmentladung. Elektrische Messungen des Zündprozesses zeigen eine nötige Zündspannung von etwa \qty{500}{V}.\\

Während die Anwendbarkeit dieser Methode zur Messung der ESEEC bestätigt werden konnte, wären zu einem breiteren Blick auf die Ergebnisse noch Messungen an weiteren Kathodenmaterialien und Arbeitsgasen notwendig.