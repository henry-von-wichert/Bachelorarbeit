\chapter{Einleitung}

Plasmen dienen in vielen Bereichen der Wissenschaft und Technik nicht nur als Versuchsobjekt, sondern auch als Werkzeug. Besonders nützlich sind dabei die hohe Elektronentemperatur bei niedriger Gastemperatur in nicht-thermischen Plasmen, sowie die Möglichkeit, Ladungsträger im Plasma gerichtet zu beschleunigen \cite{liebermanPrinciplesPlasmaDischarges1994a,pielPlasmaPhysicsIntroduction2010,chenIntroductionPlasmaPhysics1984}. Die hohe Elektronentemperatur kann in plasmachemischen Verfahren zur effizienten Überwindung der Aktivierungsenergie einer Reaktion genutzt werden, während die gerichtete Beschleunigung zum Beispiel in der Oberflächenbearbeitung genutzt wird. So werden beim Sputtern Atome von einem Target durch die Ionen des Plasmas abgetragen und ins Plasma eingebracht. Diese können dann im Plasma ionisiert und schließlich auf eine Oberfläche aufgebracht werden. So können die Eigenschaften des Plasmas unter anderem zur Herstellung schützender Schichten auf Werkzeugen, antireflektiver Schichten auf Glasscheiben und funktionaler Schichten auf Solarzellen verwendet werden \cite{weltmannFuturePlasmaScience2019,martinHandbookDepositionTechnologies2009}. Solche Methoden benötigen typischerweise einen Niederdruck in der Größenordnung von wenigen Millibar, weshalb Vakuumkammern für diese Prozesse notwendig sind. Zur Einsparung dieser Vakuumanlagen und dadurch einfacheren Einbindnung in industrielle Prozesse ist die Verwendung von Atmosphärendruckplasmen für die Prozesstechnik interessant. Auch zur Erreichung höherer Durchsätze in plasmachemischen Prozessen sind die höheren Dichten bei Atmosphärendruck nützlich. Schon seit einiger Zeit wurden dafür geeignete Plasmaquellen entwickelt \cite{winterAtmosphericPressurePlasma2015,bruggemanFoundationsAtmosphericPressure2017,tenderoAtmosphericPressurePlasmas2006}. Mit dieser Technik lassen sich nicht nur bereits bekannte Verfahren vereinfachen, sondern auch neue Prozesse ermöglichen. Ein vielversprechendes Beispiel dafür ist die medizinische Behandlung von Wunden durch Plasmen \cite{vonwoedtkePlasmasMedicine2013,gravesLowTemperaturePlasma2014a}. Plasmen wirken dabei antibakteriell und desinfizierend \cite{schneiderRoleVUVRadiation2012}.\\

Wegen des Paschen-Gesetzes sind zum Zünden von Atmosphärendruckplasmen hohe Spannungen nötig. Daher erfordert der stabile Betrieb einer solchen Entladung andere Ansätze \cite{bruggemanFoundationsAtmosphericPressure2017}. Einer dieser Ansätze ist die Miniaturisierung der Entladung in einem Mikroplasma \cite{bruggemanAtmosphericPressureDischarge2013}. Solche Entladungen ermöglichen die Zündung einer relativ kalten Glimmentladung bei Elektrodenabständen um die \qty{100}{\um}. Eine solche Entladung wird in dieser Arbeit untersucht.\\

In der Forschung an Atmosphärendruckplasmen gibt es noch viele offene Fragen. So sind bei bestimmten Prozessen Modellrechnungen nur bei Niederdruck möglich, da es bei Normaldruck deutlich mehr Stöße mit Gasteilchen gibt, die in Modellen nur schwer zu berücksichtigen sind. Einer dieser Prozesse ist die Sekundärelektronenemission, also das Phänomen, dass energiereiche Teilchen beim Aufprall auf eine Oberfläche Elektronen aus dieser herausschlagen können \cite{arumugamEffectiveSecondaryElectron2017}. Ziel dieser Arbeit ist es, durch Energiestrommessungen an einer Mikroentladung das Maß zu bestimmen, in dem die Emissionsrate der Sekundärelektronen mit dem Strom der Ionen auf die Kathode zusammenhängt.