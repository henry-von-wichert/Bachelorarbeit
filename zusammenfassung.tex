\section*{Zusammenfassung}
In dieser Arbeit wurde die Sekundärelektronenemission an einer DC-Mikroentladung untersucht. Dafür wurden Energiestrom- und elektrische Messungen durchgeführt, aus denen unter bestimmten Annahmen die effektiven Sekundärelektronenemissionskoeffizienten bestimmt werden konnten. Diese Koeffizienten setzen die Emissionsrate der Sekundärelektronen mit dem Ionenstrom auf eine Oberfläche in Verhältnis.\\

Zu der Bestimmung dieser Koeffizienten müssen verschiedene Leistungen im Plasma zusammen mit der Energieerhaltung betrachtet werden. Das Plasma wurde dabei direkt zwischen zwei passiven Thermosonden als Elektroden gezündet, was eine direkte Messung des Energiestroms auf Kathode und Anode ermöglichte. Diese Messungen wurden mit Helium als Arbeitsgas und Kupfer, Edelstahl und Tantal als Elektrodenmaterialien durchgeführt, wobei Se\-kun\-där\-el\-ek\-tro\-nen\-e\-miss\-ions\-ko\-ef\-fi\-zien\-ten der Größenordnung 1 bestimmt wurden.\\

Da die PTPs als Elektroden das Plasma nahezu vollständig unschließen, konnte ein hoher Wirkungsgrad von etwa 90\% erreicht werden. Dies ermöglicht aussagekräftigere thermische Messungen, da fast der vollständige Energiestrom aus dem Plasma aufgezeichnet wird. Da bei mangelhafter Gasreinheit Instabilitäten im Plasma auftreten, wurden sie und ihre Erhaltung im Messprozess mittels optischer Emissionsspektroskopie untersucht und bestätigt.