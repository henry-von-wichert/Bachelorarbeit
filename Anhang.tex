\chapter{Anhang}

\section{Vergleich der Energieströme verschiedener Materialien}

\begin{figure}[H]
	\centering
	\includesvg[width=0.85\linewidth]{plots/energiestrom700}
	\caption{Der Energiestrom auf beide Elektroden bei verschiedenen Materialien und \qty{700}{V} Netzteilspannung.}
	\label{fig:energiestrom}
\end{figure}
\begin{figure}[H]
	\centering
	\includesvg[width=0.85\linewidth]{plots/energiestrom800}
	\caption{Der Energiestrom auf beide Elektroden bei verschiedenen Materialien und \qty{800}{V} Netzteilspannung.}
	\label{fig:energiestrom}
\end{figure}
\begin{figure}[H]
	\centering
	\includesvg[width=0.85\linewidth]{plots/energiestrom900}
	\caption{Der Energiestrom auf beide Elektroden bei verschiedenen Materialien und \qty{900}{V} Netzteilspannung.}
	\label{fig:energiestrom}
\end{figure}
\begin{figure}[H]
	\centering
	\includesvg[width=0.85\linewidth]{plots/energiestrom1000}
	\caption{Der Energiestrom auf beide Elektroden bei verschiedenen Materialien und \qty{1000}{V} Netzteilspannung.}
	\label{fig:energiestrom}
\end{figure}

\section{Energieströme eines Materials mit verschiedenen Spannungen}

\begin{figure}[H]
	\centering
	\includesvg[width=\linewidth]{plots/einzelleistungenKupfer}
	\caption{Der Energiestrom auf Kupferelektroden in Helium bei verschiedenen Leistungen im Plasma.}
	\label{fig:einzelleistungen}
\end{figure}
\begin{figure}[H]
	\centering
	\includesvg[width=\linewidth]{plots/einzelleistungenTantal}
	\caption{Der Energiestrom auf Tantalelektroden in Helium bei verschiedenen Leistungen im Plasma.}
	\label{fig:einzelleistungen}
\end{figure}
\begin{figure}[H]
	\centering
	\includesvg[width=\linewidth]{plots/einzelleistungenEdelstahl}
	\caption{Der Energiestrom auf Edelstahlelektroden in Helium bei verschiedenen Leistungen im Plasma.}
	\label{fig:einzelleistungen}
\end{figure}